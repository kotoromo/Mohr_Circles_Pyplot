\documentclass{article}
\usepackage[utf8]{inputenc}
\usepackage[spanish]{babel}
\usepackage{bm}

\title{Círculos de Mohr}
\author{García Fierros Nicky}


\begin{document}
    \maketitle
    \pagenumbering{gobble}
    \newpage
    \pagenumbering{arabic}
    
    \tableofcontents
    \newpage

    \section{Introducción}

    El esfuerzo es una medida de \textit{intensidad de fuerza} y es resultado de un 
    cuerpo sujeto a cargas, ya sean internas o externas. En la teoría de la mecánica
    del medio continuo un cuerpo es considerado libre de esfuerzos si las únicas
    fuerzas presentes son aquellas fuerzas internas que mantienen al cuerpo unido.
    Por lo tanto, los esfuerzos que son de interés son aquellos que resultan de la
    aplicación de fuerzas por un agente externo.\\

    Los esfuerzos pueden ser transformados para ser analizados desde distintos ángulos.
    Particularmente, existe una rotación específica que permite reducir un tensor de
    esfuerzos en sólo sus componentes normales de tal manera que el análisis se
    simplifica bastánte al reducir el tensor de 9 posibles componentes a tan solo 3, de
    tal manera que el uso de las ecuaciones que involucran a dichos esfuerzos de
    simplifican bastante.\\
    
    Los círculos de Mohr son una representación gráfica de los esfuerzos normales y
    tangenciales máximos de un tensor de esfuerzos. Para obtener este diagrama, se
    hacen pasar por un punto \textit{p}, \textit{n} planos \bm{$\eta$} sobre
    los que actúan los esfuerzos de tal manera que se crea un perfil en el que
    las componentes normales y tangenciales del vector de esfuerzos \bm{$t^{\eta}$} generan círculos
    representativos de los esfuezos que se aplican sobre dicho punto referidos a los
    planos \bm{$\eta$} que se hacen pasar por este. Recuerde que, como vivimos en
    un espacio con altura, anchura y profundidad, entonces en realidad los esfuerzos
    deben ser considerados como puntos ${t}_{N}$ y ${t}_{S}$ con componentes
    orientados hacia \bm{${\eta}$}.\\

    El proyecto presentado en este documento es el de un software que grafica en un plano
    ${t}_{N}$ \textit{vs} ${t}_{S}$ los círculos de Mohr que forman un tensor de esfuerzos.
    A lo largo del documento se elaboran los análisis realizados para llegar a las 
    ecuaciones que describen a dichos círculos.

    \subsection{Definicion de esfuerzo}
    \section{Principio de Euler-Cauchy}
    \section{Tensor de esfuerzos}
    \subsection{Que es un tensor?}
    \subsection{Ecuaciones de equilibrio de la teoria de la elasticidad lineal}
    \subsection{Simetria del tensor de esfuerzos}
    \section{círculos de Mohr}
    \subsection{Analisis para generar los círculos de Mohr}
    \subsection{Esfuerzos principales}


\end{document}