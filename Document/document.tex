\documentclass{article}
\usepackage[utf8]{inputenc}
\usepackage[spanish]{babel}
\usepackage{bm}
\usepackage{amsmath}
\usepackage{esvect}
\usepackage{amssymb}

\title{Círculos de Mohr}
\author{García Fierros Nicky}


\begin{document}
    \maketitle
    \pagenumbering{gobble}
    \newpage
    \pagenumbering{arabic}
    
    \tableofcontents
    \newpage

    \pagenumbering{roman}
    \section*{Notación}

    En este documento, los vectores están expresados con letras \textbf{negritas}, con
    letras en \textit{cursiva} acompañanadas de un subíndice y un vector unitario o con
    una flecha encima.

    \begin{equation*}
        \vec{v} \Longleftrightarrow \bm{v} \Longleftrightarrow v_{i}\bm{i} \Longleftrightarrow v_{i}\vv{i_{i}}
    \end{equation*}

    \newpage
    \pagenumbering{arabic}

    \section{Introducción}

    El esfuerzo es una medida de \textit{intensidad de fuerza} y es resultado de un 
    cuerpo sujeto a cargas, ya sean internas o externas. En la teoría de la mecánica
    del medio continuo un cuerpo es considerado libre de esfuerzos si las únicas
    fuerzas presentes son aquellas fuerzas internas que mantienen al cuerpo unido.
    Por lo tanto, los esfuerzos que son de interés son aquellos que resultan de la
    aplicación de fuerzas por un agente externo.\\

    Los esfuerzos pueden ser transformados para ser analizados desde distintos ángulos.
    Particularmente, existe una rotación específica que permite reducir un tensor de
    esfuerzos en sólo sus componentes normales de tal manera que el análisis se
    simplifica bastánte al reducir el tensor de 9 posibles componentes a tan solo 3, de
    tal manera que el uso de las ecuaciones que involucran a dichos esfuerzos se
    simplifican bastante.\\
    
    Los círculos de Mohr son una representación gráfica del estado de esfuerzos en un punto 
    \textit{P}. Para obtener este diagrama, se hacen pasar por un punto \textit{P}, 
    \textit{n} planos \bm{$\eta$} sobre los que actúan los esfuerzos de tal manera que se 
    crea un perfil en el que las componentes normales y tangenciales del vector de esfuerzos 
    \bm{$t^{\eta}$} generan círculos representativos de los esfuezos que se aplican sobre 
    dicho punto referidos a los planos \bm{$\eta$} que se hacen pasar por este.\\

    El proyecto presentado en este documento es el de un software que grafica en un plano
    ${t}_{N}$ \textit{vs} ${t}_{S}$ los círculos de Mohr que caracterizan a un tensor 
    de esfuerzos. A lo largo del documento se elaboran los análisis realizados para llegar 
    a las ecuaciones que describen a dichos círculos.

    \section{Teor\'{i}a de esfuerzos}

    El esfuerzo es una medida de intensidad de fuerza producto de la aplicación de cargas
    sobre el cuerpo; ya sea sobre cada unidad de volúmen del cuerpo (fuerzas de cuerpo);
    o sobre la superficie de este; fuerzas distribuidas sobre cada unidad de superficie
    del cuerpo (fuerzas de superficie) usualmente resultado de fuerzas de contacto.

    \subsection{El principio de Euler-Cauchy}

    A partir de lo anterior, si se considera un cuerpo material homogeneo e isotrópico
    \textit{B} con superficie \textit{S}, y un volúmen \textit{V}, sujeto a fuerzas de
    cuerpo \textbf{b} y fuerzas de superficie \textbf{p} arbitrarias; un punto interno 
    al cuerpo \textit{P} por el que pasa un plano $S^{*}$ tal que divide al cuerpo
    en dos partes I y II; entonces se tiene algo como muestra la figura 1.

    Las fuerzas internas transmitidas a lo largo del plano de corte por la acción de
    la sección II sobre la sección I se distribuyen sobre el elemento de área
    $\Delta S^{*}$ que contiene al punto \textit{P} de forma que se genera una fuerza 
    resultante $\Delta$\bm{$f$} y un momento resultante $\Delta$\bm{$M$} en el punto 
    \textit{P}. El principio de Euler-Cauchy dice que en el limite conforme el área
    $\Delta S^{*}$ disminuye a cero con \textit{P} contenido en dicha área, se obtiene

    \begin{equation}
        \lim_{\Delta S^{*} \to 0} \frac{\Delta \bm{f}}{\Delta S^{*}} =
        \frac{d \bm{f}}{dS^{*}} =
        \bm{t^{\bm{(\eta)}}},
    \end{equation}

    \begin{equation}
        \lim_{\Delta S^{*} \to 0} \frac{\Delta \bm{M}}{\Delta S^{*}} = 0;
    \end{equation}

    donde el vector $\bm{t^{\bm{(\eta)}}}$ es conocido como el \textit{vector de esfuerzos}
    referido a el plano $\bm{(\eta)}$.
    Por lo tanto, para la infinidad de planos de corte $\bm{(\eta)}$ que pasan por \textit{P},
    se tienen una infinidad de vectores de esfuerzos $\bm{t^{\bm{(\eta)}}}$ para una carga
    dada aplicada al cuerpo. La totalidad de los pares $\bm{t^{\bm{(\eta)}}}$ y $\bm{(\eta)}$
    en P definen el estado de esfuerzos en \textit{P}.

    Es posible aplicar la tercera ley de Newton para demostrar que

    \begin{equation}
        \bm{t^{\bm{(\eta)}}} = - \bm{t^{\bm{(-\eta)}}},
    \end{equation}

    \begin{equation}
        -\bm{t^{\bm{(\eta)}}} = \bm{t^{\bm{(-\eta)}}}.
    \end{equation}

    \subsection{Tensor de esfuerzos}

    Recuerde que el vector de esfuerzos $\bm{t^{\bm{(\eta)}}}$ est\'{a} asociado a un plano 
    representado por su vector unitario $\bm{(\eta)}$. A partir de esta idea, se pueden
    expresar entonces los esfuerzos en el plano cartesiano con coordenadas $x_{1}, x_{2}, x_{3}$ 
    con vectores unitarios $\vec{i_{1}}, \vec{i_{2}}, \vec{i_{3}}$ de tal manera que se tiene que

    \begin{equation}
        \bm{t^{(\vec{i_{1}})}} = t_{1}^{(\vec{i_{1}})}\vec{i_{1}} + 
        t_{2}^{(\vec{i_{1}})}\vec{i_{2}} + 
        t_{3}^{(\vec{i_{1}})}\vec{i_{3}},
    \end{equation}

    \begin{equation}
        \bm{t^{(\vec{i_{2}})}} = t_{1}^{(\vec{i_{2}})}\vec{i_{1}} + 
        t_{2}^{(\vec{i_{2}})}\vec{i_{2}} + 
        t_{3}^{(\vec{i_{2}})}\vec{i_{3}},
    \end{equation}

    \begin{equation}
        \bm{t^{(\vec{i_{3}})}} = t_{1}^{(\vec{i_{3}})}\vec{i_{1}} + 
        t_{2}^{(\vec{i_{3}})}\vec{i_{2}} + 
        t_{3}^{(\vec{i_{3}})}\vec{i_{3}},
    \end{equation}

    ecuaciones que pueden ser expresadas en notaci\'{o}n \'{i}ndice como

    \begin{equation}
        \vec{t}^{(\vec{i_{i}})} = t_{j}^{(\vec{i_{i}})}\vec{i_{j}},
    \end{equation}

    \begin{equation}
        \therefore \sigma_{ij} = t_{j}^{(\vec{i_{i}})} \Longrightarrow 
        \sigma_{ij}\vec{i_{j}} = t_{j}^{(\vec{i_{i}})}\vec{i_{j}}.
    \end{equation}

    La ecuaci\'{o}n anterior puede ser escrita en notaci\'{o}n matricial de la siguiente manera:

    \begin{equation}
        \begin{bmatrix}
            \vec{t}^{(\vec{i_{1}})} \\
            \vec{t}^{(\vec{i_{2}})} \\
            \vec{t}^{(\vec{i_{3}})}
        \end{bmatrix}
        =
        \begin{bmatrix}
            \sigma_{11} && \sigma_{12} && \sigma_{13} \\
            \sigma_{21} && \sigma_{22} && \sigma_{23} \\
            \sigma_{31} && \sigma_{32} && \sigma_{33}
        \end{bmatrix}
        \begin{bmatrix}
            \vec{i_{1}} \\
            \vec{i_{2}} \\
            \vec{i_{3}}.
        \end{bmatrix}
    \end{equation}

    Adem\'{a}s es posible demostrar, dado

    %figura va aqui

    que

    \begin{equation}
        \vec{t}^{(\vec{\eta})} = \vec{\eta}\bm{\sigma}
    \end{equation}

    %\subsection{Ecuaciones de equilibrio de la teor\'{i}a de la elasticidad lineal}

    Si se considera de nuevo el estado estacionario y los momentos generados por las fuerzas
    que act\'{u}an sobre la superficie \textit{S} es posible demostrar que

    \begin{equation}
        \vec{\eta}\bm{\sigma} = \bm{\sigma}\vec{\eta}
    \end{equation}


    %\subsection{Simetr\'{i}a del tensor de esfuerzos}
    \section{Círculos de Mohr}
    \subsection{An\'{a}lisis para generar los círculos de Mohr}
    \subsection{Esfuerzos principales}


\end{document}