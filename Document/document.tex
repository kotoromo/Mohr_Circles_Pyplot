\documentclass{article}
\usepackage[utf8]{inputenc}
\usepackage[spanish]{babel}
\usepackage{bm}
\usepackage{amsmath}
\usepackage{esvect}
\usepackage{amssymb}
\usepackage{siunitx}
\usepackage{gensymb}
\usepackage{oz}
\usepackage{hyperref}
\usepackage{float}
\usepackage{graphicx}
\usepackage{subcaption}
\hypersetup{
    colorlinks=true,
    linkcolor=blue,
    urlcolor=magenta,
}

\title{Círculos de Mohr}
\author{García Fierros Nicky}


\begin{document}
    \maketitle
    \pagenumbering{gobble}
    \newpage
    \pagenumbering{arabic}
    
    \tableofcontents
    \newpage

    \pagenumbering{roman}
    \section*{Notación}

    En este documento, los vectores están expresados con letras \textbf{negritas}, con
    letras en \textit{cursiva} acompañanadas de un subíndice y un vector unitario o con
    una flecha encima.

    \begin{equation*}
        \vec{v} \Longleftrightarrow \bm{v} \Longleftrightarrow v_{i}\bm{i} \Longleftrightarrow v_{i}\vv{i_{i}}
    \end{equation*}

    \section*{Software}

    El software programado para este proyecto se encuentra programado utilizando el lenguaje de programaci\'{o}n
    \textit{Python} en su versi\'{o}n 3.6, disponible para su descarga en \url{https://www.python.org/}. Adem\'{a}s,
    los paquetes \href{https://matplotlib.org/#}{\textit{matplotlib}}, \href{http://www.numpy.org/}{\textit{numpy}} y 
    \href{https://wiki.python.org/moin/TkInter}{\textit{tkinter}} fueron empleados.

    El c\'{o}digo fuente est\'{a} disponible para su descarga en \url{https://github.com/kotoromo/Mohr_Circles_Pyplot}.

    \newpage
    \pagenumbering{arabic}

    \section{Introducción}

    El esfuerzo es una medida de \textit{intensidad de fuerza} y es resultado de un 
    cuerpo sujeto a cargas, ya sean internas o externas. En la teoría de la mecánica
    del medio continuo un cuerpo es considerado libre de esfuerzos si las únicas
    fuerzas presentes son aquellas fuerzas internas que mantienen al cuerpo unido.
    Por lo tanto, los esfuerzos que son de interés son aquellos que resultan de la
    aplicación de fuerzas por un agente externo.\\

    Los esfuerzos pueden ser transformados para ser analizados desde distintos ángulos.
    Existe una transformaci\'{o}n específica que permite reducir un tensor de
    esfuerzos en esfuerzos principales de tal manera que se logra reducir el tensor de 9 
    posibles componentes a tan solo 3, de modo que el uso de las ecuaciones que involucran
    a dichos esfuerzos se simplifican bastante.\\
    
    Los círculos de Mohr son una representación gráfica del estado de esfuerzos en un punto 
    \textit{P}. Para obtener este diagrama, se hacen pasar por un punto \textit{P}, 
    \textit{n} planos \bm{$\eta$} sobre los que actúan los esfuerzos de tal manera que se 
    crea un perfil en el que las componentes normales y tangenciales del vector de esfuerzos 
    \bm{$t^{\eta}$} generan círculos representativos de los esfuezos que se aplican sobre 
    dicho punto referidos a los planos \bm{$\eta$} que se hacen pasar por este.\\

    El proyecto presentado en este documento es un software que grafica en un plano
    ${t}_{N}$ \textit{vs} ${t}_{S}$ los círculos de Mohr que caracterizan a un tensor 
    de esfuerzos. A lo largo del documento se elaboran los análisis realizados para llegar 
    a las ecuaciones que describen a dichos círculos.

    \section{Teor\'{i}a de esfuerzos}

    El esfuerzo es una medida de intensidad de fuerza producto de la aplicación de cargas
    sobre el cuerpo; ya sea sobre cada unidad de volúmen del cuerpo (fuerzas de cuerpo);
    o sobre la superficie de este; fuerzas distribuidas sobre cada unidad de superficie
    del cuerpo (fuerzas de superficie) usualmente un resultado de fuerzas de contacto.

    \subsection{El principio de Euler-Cauchy}

    A partir de lo anterior, si se considera un cuerpo material homogeneo e isotrópico
    \textit{B} con superficie \textit{S}, y un volúmen \textit{V}, sujeto a fuerzas de
    cuerpo \textbf{b} y fuerzas de superficie \textbf{p} arbitrarias; un punto interno 
    al cuerpo \textit{P} por el que pasa un plano $S^{*}$ tal que divide al cuerpo
    en dos partes I y II. (Vease la Figura \ref{fig:cauchy})

    %Figura aqui
    \begin{figure}[H]
        \centering
        \includegraphics[width=\linewidth]{./img/cauchy}
        \caption{Fuerza y momento actuando en el punto P dentro del elemento de superficie $\Delta S^*$}
        \label{fig:cauchy}
    \end{figure}

    Las fuerzas internas transmitidas a lo largo del plano de corte por la acción de
    la sección II sobre la sección I se distribuyen sobre el elemento de área
    $\Delta S^{*}$ que contiene al punto \textit{P} de forma que se genera una fuerza 
    resultante $\Delta$\bm{$f$} y un momento resultante $\Delta$\bm{$M$} en el punto 
    \textit{P}. El principio de Euler-Cauchy dice que en el limite conforme el área
    $\Delta S^{*}$ disminuye a cero con \textit{P} contenido en dicha área, se obtiene

    \begin{equation}
        \lim_{\Delta S^{*} \to 0} \frac{\Delta \bm{f}}{\Delta S^{*}} =
        \frac{d \bm{f}}{dS^{*}} =
        \bm{t^{\bm{(\eta)}}},
    \end{equation}

    \begin{equation}
        \lim_{\Delta S^{*} \to 0} \frac{\Delta \bm{M}}{\Delta S^{*}} = 0;
    \end{equation}

    donde el vector $\bm{t^{\bm{(\eta)}}}$ es conocido como el \textit{vector de esfuerzos}
    referido a el plano $\bm{(\eta)}$.
    Por lo tanto, para la infinidad de planos de corte $\bm{(\eta)}$ que pasan por \textit{P},
    se tienen una infinidad de vectores de esfuerzos $\bm{t^{\bm{(\eta)}}}$ para una carga
    dada aplicada al cuerpo. La totalidad de los pares $\bm{t^{\bm{(\eta)}}}$ y $\bm{(\eta)}$
    en P definen el estado de esfuerzos en \textit{P}.

    Es posible aplicar la tercera ley de Newton para demostrar que

    \begin{equation}
        \bm{t^{\bm{(\eta)}}} = - \bm{t^{\bm{(-\eta)}}},
    \end{equation}

    \begin{equation}
        -\bm{t^{\bm{(\eta)}}} = \bm{t^{\bm{(-\eta)}}}.
    \end{equation}

    \subsection{Tensor de esfuerzos}

    Recuerde que el vector de esfuerzos $\bm{t^{\bm{(\eta)}}}$ est\'{a} asociado a un plano 
    representado por su vector unitario $\bm{(\eta)}$. A partir de esta idea, se pueden
    expresar entonces los esfuerzos en el plano cartesiano con coordenadas $x_{1}, x_{2}, x_{3}$ 
    con vectores unitarios $\vec{i_{1}}, \vec{i_{2}}, \vec{i_{3}}$ de tal manera que se tiene que

    \begin{equation}
        \bm{t^{(\vec{i_{1}})}} = t_{1}^{(\vec{i_{1}})}\vec{i_{1}} + 
        t_{2}^{(\vec{i_{1}})}\vec{i_{2}} + 
        t_{3}^{(\vec{i_{1}})}\vec{i_{3}},
    \end{equation}

    \begin{equation}
        \bm{t^{(\vec{i_{2}})}} = t_{1}^{(\vec{i_{2}})}\vec{i_{1}} + 
        t_{2}^{(\vec{i_{2}})}\vec{i_{2}} + 
        t_{3}^{(\vec{i_{2}})}\vec{i_{3}},
    \end{equation}

    \begin{equation}
        \bm{t^{(\vec{i_{3}})}} = t_{1}^{(\vec{i_{3}})}\vec{i_{1}} + 
        t_{2}^{(\vec{i_{3}})}\vec{i_{2}} + 
        t_{3}^{(\vec{i_{3}})}\vec{i_{3}},
    \end{equation}

    ecuaciones que pueden ser expresadas en notaci\'{o}n \'{i}ndice como

    \begin{equation}
        \vec{t}^{(\vec{i_{i}})} = t_{j}^{(\vec{i_{i}})}\vec{i_{j}},
    \end{equation}

    \begin{equation}
        \therefore \sigma_{ij} = t_{j}^{(\vec{i_{i}})} \Longrightarrow 
        \sigma_{ij}\vec{i_{j}} = t_{j}^{(\vec{i_{i}})}\vec{i_{j}}.
    \end{equation}

    %spatial figure here
    \begin{figure}[H]
        
        \centering

        \includegraphics[width=\linewidth]{./img/espacio}
        \caption{Representaci\'{o}n en el espacio de los esfuerzos.}
        \label{fig:espacio}
    \end{figure}

    La ecuaci\'{o}n anterior puede ser escrita en notaci\'{o}n matricial de la siguiente manera:

    \begin{equation}
        \begin{bmatrix}
            \vec{t}^{(\vec{i_{1}})} \\
            \vec{t}^{(\vec{i_{2}})} \\
            \vec{t}^{(\vec{i_{3}})}
        \end{bmatrix}
        =
        \begin{bmatrix}
            \sigma_{11} && \sigma_{12} && \sigma_{13} \\
            \sigma_{21} && \sigma_{22} && \sigma_{23} \\
            \sigma_{31} && \sigma_{32} && \sigma_{33}
        \end{bmatrix}
        \begin{bmatrix}
            \vec{i_{1}} \\
            \vec{i_{2}} \\
            \vec{i_{3}}.
        \end{bmatrix}
    \end{equation}
    
    \newpage

    Adem\'{a}s es posible demostrar, dado

    %figura va aqui
    \begin{figure}[H]
        
        \centering
        \includegraphics[width=\linewidth]{./img/tetrahedro}
        \caption{Tetrahedro con un v\'{e}rtice en $P$.}
        \label{fig:tetrahedro}

    \end{figure}

    que

    \begin{equation}
        \vec{t}^{(\vec{\eta})} = \vec{\eta}\cdot\bm{\sigma}
    \end{equation}

    %\subsection{Ecuaciones de equilibrio de la teor\'{i}a de la elasticidad lineal}

    Si se considera de nuevo el estado estacionario y los momentos generados por las fuerzas
    que act\'{u}an sobre la superficie \textit{S} es posible demostrar que

    \begin{equation}
        \vec{\eta}\cdot\bm{\sigma} = \bm{\sigma}\cdot\vec{\eta}
    \end{equation}


    %\subsection{Simetr\'{i}a del tensor de esfuerzos}
    \section{Círculos de Mohr}
    Como se ha explicado en las secciones anteriores, es posible definir el estado de esfuerzos
    en un punto \textit{P} al hacer pasar una cantidad infinita de planos de corte $\eta$.
    Como es imposible hacer pasar una infinidad de planos sobre un punto \textit{P}, se recurre
    entonces a utilizar un n\'{u}mero suficientemente grande de planos para generar una
    aproximaci\'{o}n aceptable del estado de esfuerzos sobre el punto \textit{P}. Es por eso
    que para este proyecto, se ha decidido utilizar $1x10^3$ planos.

    \subsection{An\'{a}lisis para generar los círculos de Mohr}
    
    Para generar los circulos de Mohr, es necesario obtener vectores de esfuerzo asociados a
    un plano arbitrario. Este vector de esfuerzo, como ya se elabor\'{o} en secciones
    anteriores puede ser calculado mediante la expresiones siguientes:
    
    \begin{equation}
        \vec{t}^{(\vec{\eta})} = \bm{\sigma} \cdot \vec{\eta},
    \end{equation}

    \begin{equation}
        \vec{t}^{(\vec{\eta})} = \sigma_{ij}\vec{i_{i}}\otimes\vec{i_{j}}\cdot\eta_{k}\vec{i_{k}}
        \Longrightarrow
        \vec{t}^{(\vec{\eta})} = \sigma_{ik}\eta_{k}\vec{i_{i}}.
    \end{equation}

    El vector de esfuerzos est\'{a} conformado por una componente normal $t_{N}$ y una
    componente tangencial $t_{S}$. 
    
    %figura del vector de esfuerzos
    \begin{figure}
        \centering
        \includegraphics[width=\linewidth]{./img/vector_t}
        \caption{Vector de los esfuerzos $\bm{t}$ y sus componentes $t_{N}$ y $t_{S}$}
        \label{fig:vector_t}
    \end{figure}

    Para obtener dichas componentes s\'{o}lo basta
    realizar un an\'{a}lisis geom\'{e}trico sencillo como se explica a continuaci\'{o}n.
    Para obtener la componente normal, se observa de la figura anterior que $t_{N}$ es la
    proyecci\'{o}n de $\vec{t}^{(\vec{\eta})}$ sobre $\vec{\eta}$, por lo que se tiene que

    \begin{equation}
        t_{N} = \vec{t}^{(\vec{\eta})} \cdot \vec{\eta},
    \end{equation}

    \begin{equation}
        t_{N} = (\sigma_{ik}\eta_{k}\vec{i_{i}})\cdot\eta_{j}\vec{i_{j}}
        \Longrightarrow
        t_{N} = \sigma_{ik}\eta_{k}\eta_{i}.
    \end{equation}

    Adem\'{a}s, se observa que la componente tangencial puede ser obtenida empleando el
    teorema de pit\'{a}goras. Se sabe que

    \begin{equation}
        \lvert \vec{t}^{(\vec{\eta})} \rvert^2 = \vec{t}^{(\vec{\eta})} \cdot \vec{t}^{(\vec{\eta})},
    \end{equation}
    
    que desarrollando la notaci\'{o} \'{i}ndice queda expresada la anterior ecuaci\'{o}n como:

    \begin{equation}
        \lvert \vec{t}^{(\vec{\eta})} \rvert^2 = \sigma_{ik}\eta_{k}\vec{i_{i}} \cdot
        \sigma_{ml}\eta_{l}\vec{i_{m}} \Rightarrow
        \lvert \vec{t}^{(\vec{\eta})} \rvert^2 = \sigma_{mk}\sigma_{ml}\eta_{k}\eta_{l},
    \end{equation}

    por lo que entonces, de la ecuaci\'{o}n

    \begin{equation}
        \lvert \vec{t}^{(\vec{\eta})} \rvert^2 = t_{S}^2 + t_{N}^2
        \Rightarrow
        t_{S}^2 = \lvert \vec{t}^{(\vec{\eta})} \rvert^2 - t_{N}^2,
    \end{equation}

    de tal manera que al sustituir la ecuaci\'{o}n (18) en la ecuaci\'{o}n (19) se tiene que 
    $t_{S}$ puede ser calculada como

    \begin{equation}
        t_{S} = \lbrack
            \sigma_{mk}\sigma_{ml}\eta_{k}\eta_{l} 
            - (\sigma_{ik}\eta_{k}\eta_{i})^2 
        \rbrack^\frac{1}{2}.
    \end{equation}

    Es importante recordar que $\bm{\eta}$ es un vector unitario y por lo tanto debe de cumplir que
    \begin{equation}
        \bm{\eta}\cdot\bm{\eta} = \eta_{i}\eta_{i} = 1.
    \end{equation}
    Es por eso que para generar los planos en el software, es necesario que, tras asignar valores aleatorios
    dentro de un rango $[-10000, 10000]$ a cada uno de los componentes, hacer que $\bm{v}$, 
    el vector generado, sea dividido por su modulo de tal manera que se obtenga un vector unitario $\bm{\eta}$.

    \begin{equation}
        \bm{\eta} = \frac{\bm{v}}{\lvert \bm{v} \rvert}
    \end{equation}

    Al programar dichas ecuaciones en el software, para el tensor de esfuerzos

    \begin{equation*}
        \begin{bmatrix}
            1 & 1 & 0 \\
            1 & 2 & 4 \\
            0 & 4 & 2
        \end{bmatrix}
    \end{equation*}

    %pantallazo del gui al ingresar el tensor para 1000 y 10000 planos
    \begin{figure}[H]
        \centering
        \begin{subfigure}[h!]{0.4\linewidth}
            \includegraphics[width=\linewidth]{./img/gui1k}
            \caption{Graficando para 1000 planos.}
        \end{subfigure}
        \begin{subfigure}[h!]{0.4\linewidth}
            \includegraphics[width=\linewidth]{./img/gui10k}
            \caption{Graficando para 10000 planos.}
        \end{subfigure}
        \caption{Captura de pantalla del software.}
        \label{fig:software}
    \end{figure}

    se obtuvieron la siguientes gr\'{a}fica:

    %grafica aqui con 1000 y 10000 planos
    \begin{figure}[H]
        \centering
        \begin{subfigure}[h!]{0.4\linewidth}
            \includegraphics[width=\linewidth]{./img/1k}
            \caption{Estado de esfuerzos en $P$ al hacer pasar 1000 planos.}
        \end{subfigure}
        \begin{subfigure}[h!]{0.4\linewidth}
            \includegraphics[width=\linewidth]{./img/10k}
            \caption{Estado de esfuerzos en $P$ al hacer pasar 10000 planos.}
        \end{subfigure}
        \caption{Graficas obtenidas del software al hacer pasar 1000 y 10000 planos.}
        \label{fig:graficas}
    \end{figure}

    Observese que al disminuir el n\'{u}mero de planos, la gr\'{a}fica muestra mas huecos, por
    lo que el estado de esfuerzos sobre el punto \textit{P} no es tan preciso.

    \newpage

    \subsection{Esfuerzos principales}

    En la introducci\'{o}n del documento se mencion\'{o} que es posible transformar el
    tensor de esfuerzos de tal manera que este se reduzca a un tensor con esfuerzos principales.

    \begin{equation}
        \overline{\sigma} = A \sigma A^{T} = 
        \begin{bmatrix}
            \overline{\sigma}_{11} & 0 & 0\\
            0 & \overline{\sigma}_{22} & 0\\
            0 & 0 & \overline{\sigma}_{33}
        \end{bmatrix}
    \end{equation}

    donde $\overline{\sigma}_{11}$, $\overline{\sigma}_{22}$ y $\overline{\sigma}_{33}$ son los esfuerzos
    principales.
    Las ecuaciones de los c\'{i}rculos de Mohr se simplifican tal que

    \begin{equation}
        \vec{\overline{t}}^{(\vec{\overline{\eta}})} = \overline{\sigma} \cdot \vec{\overline{\eta}},
    \end{equation}

    \begin{equation}
        \vec{\overline{t}}^{(\vec{\overline{\eta}})} =
        \begin{bmatrix}
            \overline{\sigma}_{11} & 0 & 0\\
            0 & \overline{\sigma}_{22} & 0\\
            0 & 0 & \overline{\sigma}_{33}
        \end{bmatrix}
        \begin{bmatrix}
            \overline{\eta}_{1} \\
            \overline{\eta}_{2} \\
            \overline{\eta}_{3}
        \end{bmatrix},
    \end{equation}

    \begin{equation}
        \vec{\overline{t}}^{(\vec{\overline{\eta}})} =
        \overline{\sigma}_{11}\overline{\eta}_{1}\vec{i_{1}} +
        \overline{\sigma}_{22}\overline{\eta}_{2}\vec{i_{2}} +
        \overline{\sigma}_{33}\overline{\eta}_{3}\vec{i_{3}}.
    \end{equation}

    Para encontrar la componente normal $t_{N}$ se reduce entonces a

    \begin{align}
        \overline{t}_{N} &= \vec{\overline{t}}^{(\vec{\overline{\eta}})} \cdot \vec{\overline{\eta}},\\
        \overline{t}_{N} &= \overline{\sigma}_{11}\overline{\eta}_{1}^2 +
                            \overline{\sigma}_{22}\overline{\eta}_{2}^2 +
                            \overline{\sigma}_{33}\overline{\eta}_{3}^2,
    \end{align}

    mientras que hallar la componente tangencial se reduce a

    \begin{align}
        \overline{t}_{S} &= \vec{\overline{\eta}} \cdot \vec{\overline{\eta}} - \overline{t}_{N}^2, \\
        \overline{t}_{S} &= \lbrack
        \overline{\sigma}_{11}^2\overline{\eta}_{1}^2 +
        \overline{\sigma}_{22}^2\overline{\eta}_{2}^2 +
        \overline{\sigma}_{33}^2\overline{\eta}_{3}^2 -
        \overline{t}_{N}^2
        \rbrack^\frac{1}{2}
    \end{align}

    Para obtener los esfuerzos principales, se propone que $\vec{\eta}$ y $\vec{t}^{(\vec{\eta})}$ se orienten
    la misma direcci\'{o}n, lo que vuelve a $\vec{t}^{(\vec{\eta})}$ en un m\'{u}ltiplo de $\vec{\eta}$, y por
    lo tanto

    \begin{equation}
        \vec{t}^{(\vec{\eta})} = \lambda\vec{\eta}
    \end{equation}
    \begin{equation}
        \vec{t}^{(\vec{\eta})} = \lambda\eta_{i}\vec{i_{i}}
    \end{equation}

    Al igualar las ecuaciones (14) y (32) se tiene que

    \begin{align}
        \sigma_{ik}\eta_{k}\vec{i_{i}} &= \lambda\eta_{i}\vec{i_{i}}\\
        (\sigma_{ik}\eta_{k} - \lambda\eta_{i})\vec{i_{i}} &= \vec{0}\\
        (\sigma_{ik}\eta_{k} - \delta_{ik}\lambda\eta_{k})\vec{i_{i}} &= \vec{0}\\
        \lbrack ( \sigma_{ik} - \delta{ik}\lambda )\eta_{k} \rbrack\vec{i_{i}} &= \vec{0}\\
        (\sigma_{ik} - \delta_{ik}\lambda)\eta_{k} &= 0
    \end{align}

    Escrito en forma matricial:
    \begin{equation}
        \begin{bmatrix}
            (\sigma_{11} - \lambda) & \sigma_{12} & \sigma_{13}\\
            \sigma_{21} & (\sigma_{22} - \lambda) & \sigma_{23}\\
            \sigma_{31} & \sigma_{32} & (\sigma_{33} - \lambda)
        \end{bmatrix}
        \begin{bmatrix}
            \eta_{1}\\
            \eta_{2}\\
            \eta_{3}
        \end{bmatrix}
        =
        \begin{bmatrix}
            0\\
            0\\
            0
        \end{bmatrix}
    \end{equation}

    La soluci\'{o}n trivial es $\vec{\eta} = \vec{0}$, pero recuerde que $\vec{\eta}$ debe ser 
    un vector unitario $\vec{\eta}\cdot\vec{\eta}=1$ y $\vec{0}$ no es unitario, por lo que
    para encontrar una soluci\'{o}n no trivial y correcta se debe evaluar el determinante de
    la matriz de coeficientes

    \begin{equation}
        \lambda^3 - I\lambda^2 + II\lambda - III = 0
    \end{equation}

    donde

    \begin{align}
        I &= \sigma_{ii} = tr(\sigma)\\
        II &= \frac{1}{2}(\sigma_{ii}\sigma_{jj} - \sigma_{ij}\sigma_{ji})\\
        III &= \epsilon_{ijk}\sigma_{1i}\sigma_{2j}\sigma_{3k} = det(\sigma)
    \end{align}

    y son conocidos como los invariantes del tensor.

    Al graficar los c\'{i}rculos de Mohr, $\overline{\sigma}$ y $\sigma$ deben generar
    la m\'{i}sma gr\'{a}fica, pues representan el mismo estado de esfuerzos ejercido
    en \textit{P}. Al obtener los esfuerzos principales del tensor
    \begin{equation*}
        \sigma_{ij} = 
        \begin{bmatrix}
            6 & 4 & 0\\
            4 & 6 & 0\\
            0 & 0 & -2
        \end{bmatrix}
    \end{equation*}

    y de su forma reducida

    \begin{equation*}
        \overline{\sigma}_{ij} = 
        \begin{bmatrix}
            10 & 0 & 0\\
            0 & -2 & 0\\
            0 & 0 & 2
        \end{bmatrix}
    \end{equation*}

    \newpage

    Al comparar sus gr\'{a}ficas, se observa que efectivamente son las mismos c\'{i}rculos,
    por lo que queda comprobado que $\sigma_{ij} \Leftrightarrow \overline{\sigma}_{ij}$.

    %graficas aqui
    \begin{figure}[H]
        \centering
        \begin{subfigure}[h!]{0.4\linewidth}
            \includegraphics[width=\linewidth]{./img/non_principal}
            \caption{Estado de esfuerzos $\sigma$ en $P$ al hacer pasar 1000 planos.}
        \end{subfigure}
        \begin{subfigure}[h!]{0.4\linewidth}
            \includegraphics[width=\linewidth]{./img/principal}
            \caption{Estado de esfuerzos $\overline{\sigma}$ en $P$ al hacer pasar 10000 planos.}
        \end{subfigure}
        \caption{Graficas obtenidas del software para el tensor de esfuerzos completo y el reducido.}
        \label{fig:graficas_2}
    \end{figure}

    \section{Conclusiones}
    Se puede concluir que el vector de los esfuerzos $\vec{t}^{\vec{\eta}}$
    representa  el estado de esfuerzos sobre un punto \textit{P} al hacer pasar un plano
    $\bm{\eta}$ por este. Al hacer pasar un n\'{u}mero grande de planos por dicho punto, se puede
    definir el estado de esfuerzos sobre ese punto. Este estado de esfuerzos est\'{a} expresado
    en el tensor de esfuerzos $\bm{\sigma}$. Una forma pr\'{a}ctica de representar el estado de
    esfuerzos es mediante los c\'{i}rculos de Mohr, pues muestran para los esfuerzos normales 
    y tangenciales sus m\'{a}ximos y m\'{i}nimos, sin importar el \'{a}ngulo del plano de corte.

    Adem\'{a}s se comprob\'{o} que efectivamente existe una transformaci\'{o}n espec\'{i}fica del tensor
    de los esfuerzos que reduce a dicho tensor en sus esfuerzos principales, simplificando enormemente
    el trabajo de c\'{a}lculo y an\'{a}lisis, de manera que el estado de esfuerzos que representan 
    $\bm{\sigma}$ y $\bm{\overline{\sigma}}$ son los m\'{i}smos.

    \section{Fuentes consultadas}
    \begin{itemize}
        \item Mase, G. E, \& Mase, G. Thomas. (1992). Continuum mechanics for engineers (2ed). EEUUA, Boca Raton: CRC Press.
        \item Apuntes tomados en clase.
    \end{itemize}

    
\end{document}
